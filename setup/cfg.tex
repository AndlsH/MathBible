% !TeX encoding = UTF-8
\usepackage[left=20mm,right=20mm,top=25mm,bottom=25mm]{geometry}% 页面设置
\usepackage[xetex,CJKbookmarks,bookmarksnumbered,bookmarksopen,colorlinks,pdfauthor={eeLord},pdftitle={Math Bible},pdfsubject={Yet another math reference},pdfkeywords={mathematics, reference}]{hyperref}% 编译程序,中文书签目录,添加章节序号,书签目录展开到节标题,链接改用彩色显示(红)。在Acrobat中创建书签目录(默认)
\usepackage{amsmath,amssymb,extarrows}  %数学
\everymath{\displaystyle}  %所有行内公式改为展示档次 %符号会变大导致不协调
%\usepackage{fontspec}
%\setmainfont{宋体}
%\usepackage{times} %公式字体
%\usepackage{mathptmx}  % Times New Roman
%\bibliographystyle{gbt7714-2005}    %国标参考文献

% 解决因行内公式造成的行距问题 *************************************
\lineskiplimit=2pt  %若两行间距小于该值
\lineskip=3pt       %则插入行间空白

