% !TeX encoding = UTF-8
\ifx \allfiles \undefined \documentclass[twoside]{ctexart} % !TeX encoding = UTF-8
 \begin{document} \fi
\section{积分表}
%================================ 正文开始 ================================

\subsection{含有 $ ax+b $ 的积分}
\[
\int\ (ax+b)^n\mbox{d}x=\frac{(ax+b)^{n+1}}{a(n+1)}+C
\]


\[
\int\frac{1}{ax+b}\mbox{d}x=\frac{1}{a}\ln \left | ax+b\right|+C
\]


\[
\int\frac{x}{ax+b}\mbox{d}x=\frac{1}{a^2}(ax+b-b\ln\left |ax+b \right|) +C
\]


\[
\int\frac{x^2}{ax+b}\mbox{d}x=\frac{1}{2a^3} \left[(ax+b)^2-4b(ax+b)+2b^2 \ln\left |ax+b\right| \right]+C
\]


\[
\int\frac{1}{x(ax+b)}\mbox{d}x = -\frac{1}{b}\ln\left | \frac{ax+b}{x}\right |+C
\]


\[
\int\frac{1}{x^2(ax+b)}\mbox{d}x=\frac{a}{b^2}\ln\left |\frac{ax+b}{x}\right |-\frac{1}{bx}+C
\]



\subsection{含有 $ \sqrt{a+bx} $ 的积分}
\[
\int x\sqrt{a+bx}\mbox{d}x=\frac{2}{15b^2}(3bx-2a)(a+bx)^{\frac{3}{2}}+C
\]


\[
\int x^2\sqrt{a+bx}\mbox{d}x=\frac{2}{105b^3}(15b^2x^2-12abx+8a^2)(a+bx)^{\frac{3}{2}}+C
\]


\[
\int x^n\sqrt{a+bx}\mbox{d}x=\frac{2}{b(2n+3)}x^n(a+bx)^{\frac{3}{2}} -\frac{2na}{b(2n+3)}\int x^{n-1}\sqrt{a+bx}\mbox{d}x
\]


\[
\int\frac{\sqrt{a+bx}}{x}\mbox{d}x=2\sqrt{a+bx}+a\int\frac{1}{x\sqrt{a+bx}}\mbox{d}x
\]


\[
\int\frac{\sqrt{a+bx}}{x^n}\mbox{d}x=\frac{-1}{a(n-1)}\frac{(a+bx)^{\frac{3}{2}}}{x^{n-1}} -\frac{(2n-5)b}{2a(n-1)}\int\frac{\sqrt{a+bx}}{x^{n-1}}\mbox{d}x,n\neq 1
\]


\[
\int\frac{1}{x\sqrt{a+bx}}\mbox{d}x=\frac{1}{\sqrt{a}}\ln\left (\frac{\sqrt{a+bx} -\sqrt{a}}{\sqrt{a+bx}+\sqrt{a}}\right )+C,a>0
\]


\[
=\frac{2}{\sqrt{-a}}\arctan\sqrt\frac{a+bx}{-a} +C,a<0
\]


\[
\int\frac{1}{x^n\sqrt{a+bx}}\mbox{d}x=\frac{-1}{a(n-1)}\frac{\sqrt{a+bx}}{x^{n-1}} -\frac{(2n-3)b}{2a(n-1)}\int\frac{1}{x^{n-1}}\sqrt{a+bx}\mbox{d}x,n\neq 1
\]



\subsection{含有 $ x^2\pm\alpha^2 $ 的积分}

\[
\int\frac{1}{x^2+\alpha^2}\mbox{d}x=\frac{\arctan\dfrac{x}{\alpha}}{\alpha}+C
\]


\[
\int\frac{1}{\pm x^2\mp\alpha^2}\mbox{d}x = \frac{\ln\left(\dfrac{x\mp\alpha}{\pm x+\alpha}\right)}{2\alpha}+C
\]



\subsection{含有 $ {ax^2+b} $ 的积分}
\[
\int\frac{1}{ax^2+b}\mbox{d}x=\frac{1}{\sqrt{ab}} \arctan\frac{\sqrt{a}x}{\sqrt{b}}+C
\]



\subsection{含有 $ ax^2+bx+c\qquad(a>0) $ 的积分}
\[
\int ax^2+bx+c\mbox{d}x=\frac{ax^3}{3}+\frac{bx^2}{2}+cx+C
\]



\subsection{含有  $ \sqrt{a^2+x^2}\qquad(a>0) $ 的积分}
\[
\int\sqrt{a^2+x^2}\mbox{d}x=\frac{1}{2}x\sqrt{a^2+x^2}+\frac{1}{2}a^2\ln\left (x+\sqrt{a^2+x^2}\right )+C
\]


\[
\int x^2\sqrt{a^2+x^2}\mbox{d}x=\frac{1}{8}x(a^2+2x^2)\sqrt{a^2+x^2}-\frac{1}{8}a^4\ln\left (x+\sqrt{a^2+x^2}\right )+C
\]


\[
\int \frac{\sqrt{a^2+x^2}}{x}\mbox{d}x = \sqrt{a^2+x^2} - a\ln \left ( \frac{a+\sqrt{a^2+x^2}}{x} \right ) +C
\]


\[
\int \frac{\sqrt{a^2+x^2}}{x^2}\mbox{d}x = \ln\left ( x+\sqrt{a^2+x^2}\right ) -  \frac{\sqrt{a^2+x^2}}{x} +C
\]


\[
\int \frac{1}{\sqrt{a^2+x^2}}\mbox{d}x=\ln \left ( x+\sqrt{a^2+x^2} \right ) +C
\]


\[
\int\frac{x^2}{\sqrt{a^2+x^2}}\mbox{d}x=\frac{1}{2}x\sqrt{a^2+x^2} -\frac{1}{2}a^2\ln\left (\sqrt{a^2+x^2}+x \right )+C
\]


\[
\int\frac{1}{x\sqrt{a^2+x^2}}\mbox{d}x=\frac{1}{a}\ln\left (\frac{x}{a+\sqrt{a^2+x^2}}\right )+C
\]


\[
\int\frac{1}{x^2\sqrt{a^2+x^2}}\mbox{d}x=-\frac{\sqrt{a^2+x^2}}{a^2x}+C
\]



\subsection{含有 $ \sqrt{x^2-a^2}\qquad{(x^2>a^2)} $ 的积分}
\[
\int \frac{1}{\sqrt{x^2-a^2}}\mbox{d}x=\ln\left ( x+\sqrt{x^2-a^2} \right ) +C
\]



\subsection{含有 $ \sqrt{a^2-x^2} \qquad(a^2>x^2) $ 的积分}

\[
\int \frac{1}{\sqrt{a^2-x^2}}\mbox{d}x= \arcsin \frac{x}{a} +C = - \arccos \frac{x}{a} +C
\]


\[
\int\sqrt{a^2-x^2}\mbox{d}x=\frac{1}{2}x\sqrt{a^2-x^2} +\frac{a^2}{2}\arcsin\frac{x}{a}+C
\]


\[
\int x^2\sqrt{a^2-x^2}\mbox{d}x=\frac{1}{8}x(2x^2-a^2)\sqrt{a^2-x^2} +\frac{1}{8}a^4\arcsin\frac{x}{a}+C
\]


\[
\int\frac{\sqrt{a^2-x^2}}{x}\mbox{d}x=\sqrt{a^2-x^2} -a\ln\left (\frac{a+\sqrt{a^2-x^2}}{x}\right )+C
\]


\[
\int\frac{\sqrt{a^2-x^2}}{x^2}\mbox{d}x=-\frac{\sqrt{a^2-x^2}}{x} -\arcsin\frac{x}{a}+C
\]


\[
\int\frac{1}{x\sqrt{a^2-x^2}}\mbox{d}x=-\frac{1}{a}\ln\left (\frac{a+\sqrt{a^2-x^2}}{x}\right )+C
\]


\[
\int\frac{x^2}{\sqrt{a^2-x^2}}\mbox{d}x=-\frac{1}{2}x\sqrt{a^2-x^2}+\frac{1}{2}a^2\arcsin\frac{x}{a}+C
\]


\[
\int\frac{1}{x^2\sqrt{a^2-x^2}}\mbox{d}x=-\frac{\sqrt{a^2-x^2}}{a^2x}+C
\]



\subsection{含有 $ R=\sqrt{|a|x^2+bx+c}\qquad(a\ne0) $ 的积分}
%%%%%%%%%%%%%%%%%%%%%%%%%%%%%%%%%%%%%%%%%%%%%%%%%%%%%%%%%%%%%%%%%%%%%%%%%%%%%
\iffalse
\[
\int\frac{\mbox{d}x}{R} = \frac{1}{\sqrt{a}}\ln\left(2\sqrt{a}R+2ax+b\right)\qquad(\mbox{for }a>0)
\]

<!-- (4.1) [Abramowitz & Stegun p13 3.3.33] + verified by differentiation -->

\[
\int\frac{\mbox{d}x}{R} = \frac{1}{\sqrt{a}}\,\operatorname{arsinh}\frac{2ax+b}{\sqrt{4ac-b^2}} \qquad \mbox{(for }a>0\mbox{, }4ac-b^2>0\mbox{)}
\]

<!-- (4.2) [Abramowitz & Stegun p13 3.3.34] + verified by differentiation -->

\[
\int\frac{\mbox{d}x}{R} = \frac{1}{\sqrt{a}}\ln|2ax+b| \quad \mbox{(for }a>0\mbox{, }4ac-b^2=0\mbox{)}
\]

<!-- (4.3) [Abramowitz & Stegun p13 3.3.35] + verified by differentiation -->

\[
\int\frac{\mbox{d}x}{R} = -\frac{1}{\sqrt{-a}}\arcsin\frac{2ax+b}{\sqrt{b^2-4ac}} \qquad \mbox{(for }a<0\mbox{, }4ac-b^2<0\mbox{, }\left(2ax+b\right)<\sqrt{b^2-4ac}\mbox{)}
\]

<!-- (4.4) [Abramowitz & Stegun p13 3.3.36] + verified by differentiation -->

\[
\int\frac{\mbox{d}x}{R^3} = \frac{4ax+2b}{(4ac-b^2)R}
\]

<!-- (4.5) [need reference] - verified by differentiation + special case of 4.7 below-->

\[
\int\frac{\mbox{d}x}{R^5} = \frac{4ax+2b}{3(4ac-b^2)R}\left(\frac{1}{R^2}+\frac{8a}{4ac-b^2}\right)
\]

<!-- (4.6) [need reference] - verified by differentiation + special case of 4.7 below-->

\[
\int\frac{\mbox{d}x}{R^{2n+1}} = \frac{2}{(2n-1)(4ac-b^2)}\left[\frac{2ax+b}{R^{2n-1}}+4a(n-1)\int\frac{\mbox{d}x}{R^{2n-1}}\right]
\]

<!-- (4.7) [need reference] - verified by differentiation only -->

\[
\int\frac{x}{R}\;\mbox{d}x = \frac{R}{a}-\frac{b}{2a}\int\frac{\mbox{d}x}{R}
\]

<!-- (4.8) [Abramowitz & Stegun p13 3.3.39] + verified by differentiation -->

\[
\int\frac{x}{R^3}\;\mbox{d}x = -\frac{2bx+4c}{(4ac-b^2)R}
\]

<!-- (4.9) [need reference] - verified by differentiation only -->

\[
\int\frac{x}{R^{2n+1}}\;\mbox{d}x = -\frac{1}{(2n-1)aR^{2n-1}}-\frac{b}{2a}\int\frac{\mbox{d}x}{R^{2n+1}}
\]

<!-- (4.10) [need reference] - verified by differentiation only -->

\[
\int\frac{\mbox{d}x}{xR}=-\frac{1}{\sqrt{c}}\ln\left(\frac{2\sqrt{c}R+bx+2c}{x}\right)
\]

<!-- (4.11) [Abramowitz & Stegun p13 implied by 3.3.38 + 3.3.33] + verified by differentiation -->

\[
\int\frac{\mbox{d}x}{xR}=-\frac{1}{\sqrt{c}}\operatorname{arsinh}\left(\frac{bx+2c}{|x|\sqrt{4ac-b^2}}\right)
\]

<!-- (4.11) [Abramowitz & Stegun p13 implied by 3.3.38 + 3.3.34] + verified by differentiation -->

\fi

\subsection{含有三角函数的积分}

\[
\int\cos x\mbox{d}x=\sin x+C
\]


\[
\int\sin x\mbox{d}x= -\cos x+C
\]


\[
\int\sec^2x\mbox{d}x=\tan x+C
\]


\[
\int\csc^2x\mbox{d}x=-\cot x+C
\]


\[
\int\sec x\tan x\mbox{d}x=\sec x+C
\]


\[
\int\csc x\cot x\mbox{d}x=-\csc x+C
\]



\[
\int\tan x\mbox{d}x=-\ln{\left|\cos {x}\right|}+C=\ln{\left|\sec x\right|}+C
\]


\[
\int\cot x\mbox{d}x=\ln{\left|\sin x\right|}+C
\]


\[
\int\sec x\mbox{d}x=\ln{\left|\sec x+\tan x\right|}+C
\]


\[
\int\csc x\mbox{d}x=-\ln{\left|\csc x+\cot x\right|}+C=\ln{\left|{\tan x-\sin x\over\sin x\tan x}\right|}+C
\]




\[
\int \sin ^n x \mbox{d}x = - \frac{1}{n} \sin ^{n-1} x \cos x + \frac{n-1}{n} \int \sin ^{n-2} x \mbox{d}x +C \quad \forall n \ge 2
\]


\[
\int \sin ^2 x \mbox{d}x = \frac{x}{2}-\frac{\sin{2x}}{4} +C
\]




\[
\int \cos ^n x \mbox{d}x = \frac{1}{n} \cos ^{n-1} x \sin x + \frac{n-1}{n} \int \cos ^{n-2} x \mbox{d}x +C \quad \forall n \ge 2
\]


\[
\int \cos ^2 x \mbox{d}x = \frac{x}{2}+\frac{\sin{2x}}{4} +C
\]




\[
\int \tan ^n x \mbox{d}x = \frac{1}{n-1} \tan ^{n-1} x - \int \tan ^{n-2} x \mbox{d}x +C \quad \forall n \ge 2
\]


\[
\int \tan ^2 x \mbox{d}x = \tan x - x +C
\]




\[
\int \cot ^n x \mbox{d}x = \frac{1}{n-1} \cot ^{n-1} x - \int \cot ^{n-2} x \mbox{d}x +C \quad \forall n \ge 2
\]


\[
\int \cot ^2 x \mbox{d}x = - \cot x - x +C
\]




\[
\int \sec ^n x \mbox{d}x = \frac{1}{n-1} \sec ^{n-2} x \tan x + \frac{n-2}{n-1} \int \sec ^{n-2} x \mbox{d}x +C \quad \forall n \ge 2
\]




\[
\int \csc ^n x \mbox{d}x = - \frac{1}{n-1} \csc ^{n-2} x \cot x + \frac{n-2}{n-1} \int \csc ^{n-2} x \mbox{d}x +C \quad \forall n \ge 2
\]



\subsection{含有反三角函数的积分}

\[
\int \arcsin x \mbox{d}x = x \arcsin x + \sqrt {1 - x^2} +C
\]


\[
\int \arccos x \mbox{d}x = x \arccos x - \sqrt {1 - x^2} +C
\]


\[
\int \arctan x \mbox{d}x = x \arctan x - \ln \sqrt {1 + x^2} +C
\]

\iffalse
\[
\int \arccot x \mbox{d}x = x \arccot x + \ln \sqrt {1 + x^2} +C
\]


\[
\int \arcsec x \mbox{d}x = x \arcsec x - \sgn(x)\ln \left|x + \sqrt{x^2 - 1}\right| +C
                                = x \arcsec x + \sgn(x)\ln \left|x - \sqrt{x^2 - 1}\right| +C
\]


\[
\int \arccsc x \mbox{d}x = x \arccsc x + \sgn(x)\ln \left|x + \sqrt{x^2 - 1}\right| +C
                                = x \arccsc x - \sgn(x)\ln \left|x - \sqrt{x^2 - 1}\right| +C
\]
\fi


\subsection{含有指数函数的积分}

\[
\int e^x\mbox{d}x=e^x+C
\]


\[
\int\alpha^x\mbox{d}x=\frac{\alpha^x}{\ln\alpha}+C
\]


\[
\int xe^{ax}\mbox{d}x=\frac{1}{a^2}(ax-1)e^{ax}+C
\]


\[
\int x^ne^{ax}\mbox{d}x=\frac{1}{a}x^ne^{ax}-\frac{n}{a}\int x^{n-1}e^{ax}\mbox{d}x
\]


\[
\int e^{ax}\sin bx \mbox{d}x=\frac{e^{ax}}{a^2+b^2}(a\sin bx-b\cos bx)+C
\]


\[
\int e^{ax}\cos bx \mbox{d}x=\frac{e^{ax}}{a^2+b^2}(a\cos bx+b\sin bx)+C
\]



\subsection{含有对数函数的积分}
\[
\int\ln x\mbox{d}x = x\ln x - x + C
\]


\[
\int\log_\alpha x\mbox{d}x=\frac{1}{\ln\alpha}\left({x\ln x - x}\right)+C
\]


\[
\int x^n\ln x\mbox{d}x = \frac{x^{n+1}}{(n+1)^2}[(n+1)\ln x -1]+ C
\]


\[
\int\frac{1}{x\ln{x}}\mbox{d}x = \ln{(\ln{x})}+C
\]



\subsection{含有双曲函数的积分}

\[
\int \sinh x \mbox{d}x = \cosh x +C
\]


\[
\int \cosh x \mbox{d}x = \sinh x +C
\]


\[
\int \tanh x \mbox{d}x = \ln\left(\cosh x\right) +C
\]


\[
\int \coth x \mbox{d}x = \ln\left(\sinh x\right) +C
\]


\[
\int \mbox{sech}\ x \mbox{d}x = \arcsin\left(\tanh x\right) + C =  \arctan\left(\sinh x\right) + C 
\]


\[
\int \mbox{csch}\ x \mbox{d}x = \ln\left(\tanh {x \over2}\right) + C
\]



\subsection{定积分}
\[
\int^\infty_{-\infty}e^{-\alpha x^2}\mbox{d}x=\sqrt{\frac{\pi}{\alpha}}
\]


\[
\int_0^\frac{\pi}{2} \mbox{sin}^n x\mbox{d}x=\int_0^\frac{\pi}{2} \mbox{cos}^n x\mbox{d}x=
\begin{cases}
\frac{n-1}{n}\cdot\frac{n-3}{n-2}\cdot\ldots\cdot\frac{4}{5}\cdot\frac{2}{3}, & \mbox{if }n>1\mbox{ and }n\mbox{ is odd} \\
\frac{n-1}{n}\cdot\frac{n-3}{n-2}\cdot\ldots\cdot\frac{3}{4}\cdot\frac{1}{2}\cdot\frac{\pi}{2}, & \mbox{if }n>0\mbox{ and }n\mbox{ is even}
\end{cases}
\]




%================================ 正文结束 ================================
\ifx \allfiles \undefined \end{document} \fi