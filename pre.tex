% !TeX encoding = UTF-8
\usepackage[left=20mm,right=20mm,top=25mm,bottom=25mm]{geometry}% 页面设置
%\usepackage[xetex,CJKbookmarks,bookmarksnumbered,colorlinks]{hyperref}% 编译程序,中文书签目录,添加章节序号,链接改用彩色显示(红)。在Acrobat中创建书签目录(默认)
\usepackage{enumerate,pifont,paralist}  %排序列表,带圈
\usepackage{graphicx,float} %插图环境%固定图片位置
\usepackage{amsmath,amssymb,extarrows}  %数学
\usepackage{makecell,tablists,tabularx,tabulary}    %(类)表格相关
\everymath{\displaystyle}  %所有行内公式改为展示档次 %符号会变大导致不协调
%\usepackage{fontspec}
%\setmainfont{宋体}
%\usepackage{times} %公式字体
%\usepackage{mathptmx}  % Times New Roman
%\bibliographystyle{gbt7714-2005}    %国标参考文献

% 标题设置 ********************************************************
\ctexset{
    part/name={第,篇}
}

% 便捷输入 ********************************************************
\newcommand{\ud}{\,\mathrm{d}}      %使用 \ud输入微分号
\newcommand{\dx}{\,\mathrm{d} x}    %使用 \dx输入 dx
\newcommand{\zy}{\leftrightarrow}   %单左右箭头
\newcommand{\xo}{x_{0}}             %使用 \xo(字母 o)输入 x0
\usepackage[integrals]{wasysym}     %该命令直接修改了 \int
\DeclareMathOperator{\arccot}{arccot}   %定义算符 \arccot
%\usepackage{libertine}
%\usepackage{tikz}
%    \newcommand*\circled[1]{\tikz[baseline=(char.base)]{
%        \node[shape=circle,draw,inner sep=1pt] (char) {#1};}}

% 解决因行内公式造成的行距问题 *************************************
\lineskiplimit=1pt  %若两行间距小于该值
\lineskip=4pt       %则插入行间空白
